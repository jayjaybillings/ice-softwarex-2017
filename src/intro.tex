\section{Introduction}

High performance modeling and simulation software is hard to use. Much of the
challenge comes from the inherent nature of the science
under consideration, but perhaps just as much of the problem comes from
focusing very heavily on raw compute performance. Focusing on performance is
understandable, given the field, but it leaves the rest of us to figure out how
to actually use the software in a way that scales for a much larger set of
users, many of whom may be novices or migrate between software packages regularly.

In previous work, Billings et. al., discovered through requirements gathering
interviews that many of the difficulties using high performance modeling and
simulation software fall broadly into five distinct categories,
\cite{billings_cbhpc}. This includes input generation and preprocessing (also
called ``model setup''); job execution and monitoring; postprocessing,
visualization and data analysis; data management; and customizing the software.
There are many tools that address these problems individually, but the same
work found that the excess number and specialization of these tools also
contribute to the learning curve.

Efforts to address these five issues typically fall in with general purpose
scientific workflow tools like Kepler, \cite{kepler}, or are reduced to myopic
tools that satisfy some set of requirements for a single piece of software or
platform. That is, the proposed solution to this problem is often to shoe-horn
it into existing workflow tools that are so general that they focus on nothing
in particular or to ignore the general problem entirely and deploy a completely
tailored solution for the given application. These are opposing extremes, but a
middle-of-the-road solution is also possible. A workflow engine could be
developed that limits its scope to High-Performance Computing (HPC) and to the
set of possible workflows that come from the previously mentioned five
activities. A rich enough Application Programming Interface (API) could be
exposed so that highly customized solutions could still be made based on this
limited workflow engine with only a relatively minor amount of additional
development required.

It is not clear that one of these solutions is better than the others.
Practical requirements will ultimately dictate which way projects go. This work
considers the middle ground sollution and presents the Eclipse Integrated
Computational Environment (ICE) as proof that it is possible to create such a
system. Specifically, we show
\begin{itemize}
\item an architecture for such a workflow system that accomplishes all five of
the required activities outlined above in a simple, extensible way, \ref{}.
\item that such a system can be cross-platform for workstations and
simultaneously deployable on the web, \ref{}.
\item that smart design decisions enable not only authoring code for simulation
software, but also make it possible for the system to extend itself, thereby
enabling heavy customization, \ref{}.
\item that the system can be easily integrated with other tools, including the
general-purpose workflow engines and single-focus tools on opposite ends of this
tool space, \ref{}.
\end{itemize}
